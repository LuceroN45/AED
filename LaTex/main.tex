%%%%%%%%%%%%%%%%%%%%%%%%%%%%%%%%%%%%%%%%%%%%%%%%%%%%%%%%%%%%%%%
%
% Welcome to Overleaf --- just edit your LaTeX on the left,
% and we'll compile it for you on the right. If you open the
% 'Share' menu, you can invite other users to edit at the same
% time. See www.overleaf.com/learn for more info. Enjoy!
%
%%%%%%%%%%%%%%%%%%%%%%%%%%%%%%%%%%%%%%%%%%%%%%%%%%%%%%%%%%%%%%%
\documentclass{article}
\usepackage[utf8]{inputenc}
\usepackage[spanish]{babel}
\usepackage{listings}
\usepackage{graphicx}
\usepackage{algpseudocode} %algorithmic


\title{Clase de LaTex}
\author{Nicolas Lucero xd}

\begin{document}
\maketitle

\section*{Ejercicio 1}
\begin{center}
    ¡Llegaron los resultados de la carrera!
\end{center}

\begin{enumerate}
    \item En primer lugar, y recibiendo un premio de \$0 llega Alice.
    \item En segundo lugar, a nada de ganar el \#1 lugar , llega Bob.
    \item Por último, pero incrementando su participación en carreras desde la última temporada en un 120 \%, llega Charlie.
    \item \dots
\end{enumerate}

\section*{Prueba de ecuación}
Las raices de una ecuación cuadrática son
\begin{equation}
x = \frac{-b \pm \sqrt{b^2 - 4ac}}
{2a}
\end{equation}
donde $a$, $b$ y $c$ son \ldots

\section*{Prueba de codigo}
\begin{lstlisting}[language=c]
int sumar(int x, int y) {
return x + y;
}
\end{lstlisting}

\section*{Ejercicio 2}
(\dots) teniendo en cuenta esto, podemos definir el \textbf{teorema de Bayes}
como:
\begin{equation}
    P(A|B)=\frac{P(A|B)P(A)}{P(B)}
\end{equation}
Para utilizar esto, queremos primero implementar la función
\texttt{probarCondicional} de la siguiente manera

\begin{lstlisting}[language=python]
probaCondicional(A,B):
    proba = proba (interseccion(A,B))
    denominador = proba(B)
    return division(numerador,denominador)
\end{lstlisting}


%\section*{Imagen}
%\begin{figure}
%    \centering
%    \includegraphics{./imageUBA.png}
%    \caption{Universidad de Buenos Aires}
%    \label{fig:placeholder}
%\end{figure}



\section*{Ejercicio 3}

\begin{algorithmic}
\Function{sumartuplas}{Tuplas}
    \State $suma \leftarrow (0,0)$
    \If{$tupla \neq \{\}$}
        \For {$i = 1,\dots,|Tuplas|$}
            \State $suma_1 \leftarrow tupla[i]_1$
            \State $suma_2 \leftarrow tupla[i]_2$
        \EndFor
    \EndIf
    \State \Return $suma$
\EndFunction
\end{algorithmic}

\end{document}